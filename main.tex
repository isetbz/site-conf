% !TeX program = xelatex
\documentclass{site-conf}

\title{Your Paper Title}

\author{
  \begin{tabular}{ccc}
    \name{Author 1} & \name{Author 2} & \name{Author 3} \\
    \departement{Department Name} & \departement{Department Name} &
    \departement{Department Name}\\
    \university{University Name} & \university{University Name} &
    \university{University Name} \\
    \email{email@email.email} & \email{email@email.email} &
    \email{email@email.email}
  \end{tabular}
}

\begin{document}
\maketitle

\begin{abstract}
  Your abstract goes here.
\end{abstract}

\begin{keywords}
  keyword1, keyword2, keyword3
\end{keywords}

\section{Introduction}
Your content here... \cite{ref1} discusses the importance of... and
\cite{ref2} provides a comprehensive overview of... The following
sections are illustrative examples. Adapt them freely to suit your needs.

\section{Methodology}
Your methodology description here...

\subsection{Mathematical Expressions}
We will cover how to format mathematical expressions in \LaTeX.

\subsubsection{Inline Math}
For simple expressions within text, we use inline math. For example,
the quadratic formula is $x = \frac{-b \pm \sqrt{b^2 - 4ac}}{2a}$
where $a \neq 0$. The area of a circle with radius $r$ is $A = \pi
r^2$, and the circumference is $C = 2\pi r$. In probability theory,
we often work with expectations like $E[X] = \mu$ and variances
$\text{Var}(X) = \sigma^2$.

\subsubsection{Displayed Equations}
For more complex or important equations, we use display mode:

\[
  \int_{-\infty}^{\infty} e^{-x^2} dx = \sqrt{\pi}
\]

The famous Euler's identity:
\[
  e^{i\pi} + 1 = 0
\]

\subsubsection{Numbered Equations}
When equations need to be referenced, we use the equation environment:

\begin{equation}
  \nabla \cdot \mathbf{E} = \frac{\rho}{\epsilon_0}
  \label{eq:gauss}
\end{equation}

\begin{equation}
  \frac{\partial u}{\partial t} = \alpha \frac{\partial^2 u}{\partial x^2}
  \label{eq:heat}
\end{equation}

Eq.~\ref{eq:gauss} represents Gauss's law, while Eq.~\ref{eq:heat} is
the heat equation.

\subsubsection{Multiple Equations}
For systems of equations, we can use the align environment:

\begin{align}
  x + y &= 5 \label{eq:sys1} \\
  2x - y &= 1 \label{eq:sys2}
\end{align}

From Eq.~\ref{eq:sys1} and Eq.~\ref{eq:sys2}, we can solve for $x =
2$ and $y = 3$.

\section{Results and Discussion}
Your discussion of the results here...

As shown in Fig.~\ref{fig:sample}, our approach consists of multiple
components. The figure demonstrates the basic structure of our proposed method.

\begin{figure}[htbp]
  \centering
  \fbox{\parbox{0.8\columnwidth}{\centering\vspace{2cm}Sample Figure
  Content\vspace{2cm}}}
  \caption{Sample figure demonstrating the basic structure}
  \label{fig:sample}
\end{figure}

As shown in Tab.~\ref{tab:comparison}, Method B consistently outperforms the
alternatives.

\begin{table}[htbp]
  \centering
  \caption{Performance comparison across different datasets}
  \label{tab:comparison}
  \begin{tabular}{lccc}
    \hline
    Dataset & Method A & Method B & Method C \\
    \hline
    Dataset 1 & 0.823 & 0.891 & 0.856 \\
    Dataset 2 & 0.767 & 0.834 & 0.798 \\
    Dataset 3 & 0.902 & 0.923 & 0.911 \\
    \hline
  \end{tabular}
\end{table}

\section{Conclusion}
Your conclusion here...

\section{Acknowledgments}
Acknowledgments go here. You can thank individuals or organizations
that contributed to your work.

\begin{thebibliography}{99}
  \bibitem{ref1}
  Author Name, \emph{Title of the Paper}, Journal Name, vol. 1, no.
  1, pp. 1-10, Year.
  \bibitem{ref2}
  Author Name, \emph{Title of the Book}, Publisher, Year.
\end{thebibliography}

\end{document}
